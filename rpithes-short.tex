%%%%%%%%%%%%%%%%%%%%%%%%%%%%%%%%%%%%%%%%%%%%%%%%%%%%%%%%%%%%%%%%%%% 

%                       rpithes-short.tex                         %

%         Template for a short thesis all in one file             %

%        (titlepage info below assumes masters degree}            %

%  Just run latex (or pdflatex) on this file to see how it looks  %

%      Be sure to run twice to get correct TOC and citations      %

%%%%%%%%%%%%%%%%%%%%%%%%%%%%%%%%%%%%%%%%%%%%%%%%%%%%%%%%%%%%%%%%%%% 

%

%  To produce the abstract title page followed by the abstract,

%  see the template file, "abstitle-mas.tex"

%

%%%%%%%%%%%%%%%%%%%%%%%%%%%%%%%%%%%%%%%%%%%%%%%%%%%%%%%%%%%%%%%%%%%



\documentclass{thesis}

\usepackage{graphicx}   % if you want to include graphics files


% Use the first command below if you want captions over 1 line indented.

% A side effect of this is to remove the use of bold for captions. 

% To restore bold, also include the second line below.

%\usepackage[hang]{caption}     % to indent subsequent lines of captions

%\renewcommand{\captionfont}{\bfseries} % only needed with caption package;

                                        %   otherwise bold is default)

                                        
%%%%%%%%%%%%%%%%%%%%  supply titlepage info  %%%%%%%%%%%%%%%%%%%%%

\thesistitle{\bf Acoustics Visualization\\For Architectural Spaces}        

\author{Max Espinoza}        

\degree{Master of Science}

\department{Computer Science} % provide your area of study here; e.g.,

%  "Mechanical Engineering", "Nuclear Engineering", "Physics", etc.

% \thadviser{Dr. Barbra Cutler}
\projadviser{Dr. Barbra Cutler}

%\cothadviser{First co-adviser} %if needed

%\cocothadviser{Second co-adviser} % if needed

%  For a masters project use \projadviser instead of \thadviser, 

%  and \coprojadviser and \cocoprojadviser if needed. 

\submitdate{January 1685\\(For Graduation May 1685)}        

%\copyrightyear{1685}  % if date omitted, current year is used. 

%%%%%%%%%%%%%%%%%%%%%   end titlepage info  %%%%%%%%%%%%%%%%%%%%%%

      

\begin{document} 

\titlepage             % Print titlepage   

%\copyrightpage        % optional         

\tableofcontents       % required 

\listoftables          % required if there are tables

\listoffigures         % required if there are figures


%%%%%%%%%%%%%%%%%%%%%%%%%%%%%%%%%%%%%%%%%%%%%%%%%%%%%%%%%%%%%%%%%%%%%%%%%%%%%%%
%% ╔╦╗╔═╗╔╦╗╔═╗
%%  ║ ║ ║ ║║║ ║
%%  ╩ ╚═╝═╩╝╚═╝
%% [ ] Acoustics == Sound ?? Can I use the term interchangeably
%%
%%
%%
%%%%%%%%%%%%%%%%%%%%%%%%%%%%%%%%%%%%%%%%%%%%%%%%%%%%%%%%%%%%%%%%%%%%%%%%%%%%%%%

%%%%%%%%%%%%%%%%%%%%%%%%%%%%%%%%%%%%%%%%%%%%%%%%%%%%%%%%%%%%%%%%%%%%%%%%%%%%%%%
%% Acknowledgment                                                            %%
%%%%%%%%%%%%%%%%%%%%%%%%%%%%%%%%%%%%%%%%%%%%%%%%%%%%%%%%%%%%%%%%%%%%%%%%%%%%%%%
\specialhead{ACKNOWLEDGMENT}
The acknowledgment text goes here. Unlike chapter headings, 
this heading is not numbered.

%%%%%%%%%%%%%%%%%%%%%%%%%%%%%%%%%%%%%%%%%%%%%%%%%%%%%%%%%%%%%%%%%%%%%%%%%%%%%%%
%% Abstract                                                                  %%
%%%%%%%%%%%%%%%%%%%%%%%%%%%%%%%%%%%%%%%%%%%%%%%%%%%%%%%%%%%%%%%%%%%%%%%%%%%%%%%

\specialhead{ABSTRACT}

% One Sentence Topic (Try number 1)
Interactive visualization of sound propagation through architectural spaces 
offers designers an iterative approach to construct spaces with desired
acoustical proprieties through changes in geometry and materials within a 
simulated computer models.
% Research question
We are investigating methods to simulate and visualize sound propagations with 
interactive frame rates for new geometries with less noise then existing 
algorithms.
% FIXME Generalizing other algorithms
Simulating sound wave propagations can be done in real time with existing 
algorithms, however these algorithms require non real-time pre-computation for 
each room geometry simulated.
Other algorithms that do not require an offline computation step offer results
in real time, however yield noisy visualizations, require a large initial
sampling, or do not capture all acoustical phenomenons;
% How did we tackle this question
Extending wave particles, a concept introduced by Cem Yuksel for real time 
water simulation through hight fields, to simulate the wave front propagations 
of sound we can offer clear visualizations at interactive frame rates for an
iterative designs of room geometries of materials.
% How did we go about this
We created a design tool for architect students that will allow the 
creation of 3D models and interactive visualizations of sound wave fronts in
created spaces given an emitter and receiver. 
% Impact of research
With this toolkit designers will be able to assess the general acoustical
properties of a room in the early design phase. Furthermore we support BDRFs 
found in commercial wall materials that will allow designers to creatively use
these to achieve desired acoustical properties.


%%%%%%%%%%%%%%%%%%%%%%%%%%%%%%%%%%%%%%%%%%%%%%%%%%%%%%%%%%%%%%%%%%%%%%%%%%%%%%%
%% Introduction                                                              %%
%%%%%%%%%%%%%%%%%%%%%%%%%%%%%%%%%%%%%%%%%%%%%%%%%%%%%%%%%%%%%%%%%%%%%%%%%%%%%%%

\chapter{INTRODUCTION}
The text of the first chapter goes here. To cite a reference for the
bibliography, use a command such as: % \cite{thisbook}

\section{A Section Heading}

This is a sentence to take up space and look like text.

\subsection{A Subsection Heading}

\chapter{THE NEXT CHAPTER}

And so on, for more chapters.

Another citation for the bibliography:\cite{anotherbook}


% The following produces a numbered bibliography where the numbers

% correspond to the \cite commands in the text.

\specialhead{LITERATURE CITED}

\begin{singlespace}

\begin{thebibliography}{99}

\bibitem{thisbook} This is the first item in the Bibliography.

Let's make it very long so it takes more than one line.

Let's make it very long so it takes more than one line.

\bibitem{anotherbook} The second item in the Bibliography.

\end{thebibliography}

\end{singlespace}



%%%%%%%%%%%%%%%%%%%%%%%  For Appendices  %%%%%%%%%%%%%%%%%%%

\appendix    % This command is used only once!

\addtocontents{toc}{\parindent0pt\vskip12pt APPENDICES} %toc entry, no page #

\chapter{THIS IS AN APPENDIX}

Note the numbering of the chapter heading is changed.

This is a sentence to take up space and look like text.

\section{A Section Heading}

This is how equations are numbered in an appendix:

%\begin{equation}
%
%x^2 + y^2 = z^2
%
%\end{equation} 



\chapter{THIS IS ANOTHER APPENDIX}

This is a sentence to take up space and look like text.



\end{document}

